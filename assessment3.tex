\documentclass[a4paper,10pt]{article}
\usepackage{a4wide}
\usepackage[T1]{fontenc}
\usepackage[utf8]{inputenc}
\usepackage[english]{babel}
\usepackage{enumerate} 		%allows you to change counters
\usepackage{verbatim}

\title{
	Assignment 3: Map-Reduce in Erlang\\
	Advanced Programming
  }
\author{
	Guðmundur Páll Kjartansson\\
	Jón Gísli Egilsson	
}

% Uncomment to set paragraph indentation to 0 points, and skips a line
% after an ended paragraph.
\setlength{\parindent}{0pt}
\setlength{\parskip}{\baselineskip}

\begin{document}
\maketitle

\section{Assessment}

\subsection{Overview}
For this assignment we implemented a framework for Map-Reduce and used it for a few tasks.

We followed the design from the skeleton given to us mostly unchanged. We only added a debug parameter to \verb!gather_data_from_mappers!, which we used to print out the progress in the reducer.

Our frameworks workflow is like this:

\begin{enumerate}
\item User calls the \verb!start! function. Mappers, Reducer and Coordinator are started with no functions or input data.
\item User calls job, which does an rpc call to the Coordiantor with all initialization data and waits for an answer. 
\item Coordinator sends map function to all the mappers and spawns a node that feeds data to them with \verb!send_data!.
\item Coordinator does a rpc call to the Reducer with reducer function, initial accumulator value and datalength and waits for an answer from Reducer.
\item Reducer goes into gather data mode. It now starts receiving data asyncroniously from the Mappers and supplies it to the Reduce function.
\item When Reducer knows that all data has been received and reduced (datalength counter goes to 0), it goes back to its initial mode and replies to the Coordinator with the return value.
\item Coordinator receives return value and sends to job function
\item job function sends value back to user.
\item User goes back to step 2 or calls the stop function, which kills the Mappers, Reducer and Coordinator.
\end{enumerate}

\subsection{Quality of code}
We think our solution is simple and robust. We haven't run into any problems with our current version of the Map-Reduce framework with any of our tests. 

Some jobs naturally take longer than others. We have however not detected if there are bottlenecks as we did not measure execution time of individual parts of our framework.

We also didn't do any kind of error-handling, which would not be acceptable for a commercial product.

\subsection{Testing}


\end{document}




































